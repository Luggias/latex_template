\chapter{Operational Risk \& Resilience}\label{ch:oprisk}

\begin{abstract}
Operational failures—cyber‑attacks, system outages, third‑party breaches—
can dwarf market losses.  We cover frameworks (BCM, NIST CSF),
cyber‑defence layers, cloud risk and incident response.
\end{abstract}

\section{Risk Taxonomy}

\begin{itemize}
  \item \textbf{Technology} – software bugs, latency amplification.
  \item \textbf{People} – privilege misuse, key‑person dependency.
  \item \textbf{Process} – flawed change management, poor documentation.
\end{itemize}

\section{Business‑Continuity Management (BCM)}

* Recovery‑time objective (RTO)  
* Recovery‑point objective (RPO)  
* Tier‑0 hot‑hot datacentres with \(\le\SI{15}{\minute}\) RPO.

\section{Cyber‑Security Controls}

Zero‑trust network segmentation, MFA, SOC 2 and ISO 27001 certification,
red‑team exercises.

\section{Cloud and Third‑Party Risk}

Shared‑responsibility model; vendor lock‑in; regulatory localisation
(e.g.\ EU DORA).

\section{Incident Response Playbook}

Detection → Containment → Eradication → Recovery → Post‑mortem (blameless).

\section*{Key Takeaways}
\begin{enumerate}
  \item Resilience is architectural—cannot be patched after go‑live.
  \item Board‑level ownership and rehearsed run‑books cut downtime impact.
\end{enumerate}
