\chapter{Discretionary vs Systematic Trading Styles}\label{ch:discvsyst}

\begin{abstract}
Trading philosophies span a spectrum from human intuition to fully
automated decision‐making.  We contrast discretionary and systematic
approaches in terms of signal generation, execution, governance and
scalability.
\end{abstract}

\section{Taxonomy}

\begin{itemize}
  \item \textbf{Discretionary}: portfolio manager synthesises macro news,
        order flow, and qualitative insights to make directional bets.
  \item \textbf{Quantitative Systematic}: rules‐based models translate
        data into positions; execution often automated.
\end{itemize}

\section{Edge Identification}

Discretionary edge: information asymmetry, situational awareness.  
Systematic edge: statistical regularities, speed, breadth.

\section{Risk Governance}

Humans prone to bias and style drift; systems vulnerable to model error
and regime change.  
\(\Rightarrow\) implement \emph{kill switches}, exposure limits,
and real‑time monitoring.

\section*{Key Takeaways}

\begin{enumerate}
  \item Neither style dominates universally—hybrid teams increasingly common.
  \item Discipline and post‑trade review close the loop on both fronts.
\end{enumerate}
