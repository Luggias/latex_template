% !TeX root = ../latex_template.tex

\section{Tutorial}

\subsection{\LaTeX{} Set-Up}
\subsection{Commands}
\begin{verbatim}
  \section{}
  \subsection{}
  \subsubsection{}

  \section*{}
  \subsection*{}
  \subsubsection*{}
  \subsubsubsection{}

  \paragraph{}
  \subparagraph{}
\end{verbatim}

\section{Features}
\subsection{Tables}


\begin{table}[ht]
  \centering
  \begin{tabular} {|l|c|r|}
    \hline
    Header 1 & Header 2 & Header 3 \\
    \hline
    Row 1 Col 1 & Row 1 Col 2 & Row 1 Col 3 \\
    \hline
  \end{tabular}
  \caption{}
\end{table}

\subsection{Shortcuts}


\subsection{Snippets}

\newpage
\section{Code IDE}
\subsection{Java}
Dies ist ein \code|Java| code.
\begin{java}
public class Stock {
    private String symbol;
    private double sharePrice;

    public Stock(String sym, double price) {
        this.symbol = sym;
        this.sharePrice = price;
    }

    public Stock(String sym) {
        this(sym, 0.0); // constructor chaining
    }
}
\end{java}
\subsection{Python}
\begin{python}
    class DataAnalyzer:
    def __init__(self, data_points):
        self.data_points = data_points

    def calculate_average(self):
        """Calculates the average of the data points."""
        if not self.data_points:
            return 0
        return sum(self.data_points) / len(self.data_points)

# Example usage
analyzer = DataAnalyzer([10, 20, 30, 40, 50])
avg = analyzer.calculate_average()
print(f"The average is: {avg}")
\end{python}

öajsdföljasöl
\textbf{\underline{Example:}}
\begin{itemize}
  \item TEstjölaskdfjalsjfasöld
  \item öasldkfjölasdkjföalkdsjföadlskfj
  \item asödfkjaölsdf
\end{itemize}
ölaksdjfölkölas
\par
öajsdföljasöl \textbf{\underline{Example:}}
\begin{enumerate}
  \item TEstjölaskdfjalsjfasöld
  \item öasldkfjölasdkjföalkdsjföadlskfj
  \item asödfkjaölsdf
\end{enumerate}
ölaksdjfölkölas

\vspace{0.4cm}
\hrule
\vspace{0.4cm}

\subsection{Servus "Obst"} 
\subsubsection{Griasdi}
\subsubsubsection{Test}

\section{Test}

\begin{table}[h!]
  \centering
  \begin{tabular}{lcr}
    \toprule
    Header1 & Header2 & Header3 \\
    \midrule
    Row1 Col1 & Row1 Col2 & Row1 Col3 \\
    \bottomrule
  \end{tabular}
  % caption{}
  % label{tab:}
\end{table}

\begin{table}[h!]
  \centering
  \begin{tabular} {|l|c|r|} 
    \hline
    Header1 & Header2 & Header3 \\
    \hline
    Row1 Col1 & Row1 Col2 & Row1 Col3 \\
    \hline
  \end{tabular}
  % caption{}
  % label{tab:}
\end{table}


\begin{tcolorbox}[
  colback=blue!20,
  colframe=blue!60,
  title={Theorem 1},
  % caption={Caption text},
  % label={box:label}
]
  Lorem ipsum, ich dreh den Sack um.
\end{tcolorbox}

\begin{tcolorbox}[
  colback=green!20,
  colframe=green!40!black,
  title={Title},
  % caption={Caption text},
  % label={box:label}
]
  Lorem ipsum, ich dreh den Sack um.
\end{tcolorbox}

\begin{tcolorbox}[
  colback=red!20,
  colframe=red!80!black,
  title={Title},
  % caption={Caption text},
  % label={box:label}
]
  aslkdjfaölskdjföajdk
\end{tcolorbox}

\begin{tcolorbox}[
  colback=yellow!20,
  colframe=yellow!60!black,
  title={Title},
  % caption={Caption text},
  % label={box:label}
]
  xcvbnm,sdajfkasdf
\end{tcolorbox}

\begin{tcolorbox}[
  colback=white!10,
  colframe=blue!60,
  title={Title},
  % caption={Caption text},
  % label={box:label}
]
  asjfölkjasdlf
\end{tcolorbox}

\begin{minted}{python}
import numpy as np
from scipy.stats import norm

def bs_call(S0, K, r, sigma, T):
    d1 = (np.log(S0/K) + (r + 0.5*sigma**2)*T)/(sigma*np.sqrt(T))
    d2 = d1 - sigma*np.sqrt(T)
    return S0*norm.cdf(d1) - K*np.exp(-r*T)*norm.cdf(d2)
\end{minted}

\begin{tcolorbox}[
  colback=blue!20,
  colframe=blue!60,
  title={Title},
  % caption={Caption text},
  % label={box:label}
]
  
\end{tcolorbox}