\chapter{Greeks and Sensitivity Analysis}\label{ch:greeks}

\begin{abstract}
Greeks quantify risk in multidimensional parameter space.  We derive
analytic formulas under Black--Scholes, examine higher‑order metrics,
and show Python checks using automatic differentiation.
\end{abstract}

\section{First‑Order Greeks}

\[
\begin{aligned}
\Delta &= \partial_S C = N(d_1), \\
\Theta &= \partial_t C = -\frac{S\sigma N'(d_1)}{2\sqrt{T}}
                          - r K e^{-rT} N(d_2), \\
\rho   &= \partial_r C = K T e^{-rT} N(d_2), \\
\nu    &= \partial_\sigma C = S\sqrt{T} N'(d_1).
\end{aligned}
\]

\section{Second‑Order Greeks}

Gamma \(\Gamma = \partial^2_{SS} C = N'(d_1)/(S\sigma\sqrt{T})\).  
Vanna, vomma, speed for volatility trading desks.

\section{Python Automatic Differentiation (JAX)}

\begin{minted}{python}
import jax.numpy as jnp
from jax import grad

bs = lambda S, K, r, sigma, T: ...
delta = grad(bs, argnums=0)
gamma = grad(delta, argnums=0)
\end{minted}

\section{Greeks Aggregation and Risk Reports}

Risk P\&L:
\(\Delta P \approx \Delta\cdot\Delta S + \tfrac12\Gamma(\Delta S)^2
            + \nu\Delta\sigma + \dots\)

\section*{Key Takeaways}
\begin{itemize}
  \item Greeks are partial derivatives; numerical estimation requires
        step‑size control.
  \item Risk aggregation across portfolios uses position‑weighted sums.
\end{itemize}
