\chapter{Portfolio Management and Allocation}\label{ch:portfolio}

\begin{abstract}
We bridge option‑centric risk analytics with portfolio theory:
CAPM, multi‑factor models, risk‑parity, and Kelly sizing.
\end{abstract}

\section{CAPM Refresher}

\(E[R_i] = r_f + \beta_i(E[R_m]-r_f)\).

\section{Factor Models}

Fama–French 3‑/5‑factor, momentum, quality minus junk (QMJ).
Estimation in Python:

\begin{minted}{python}
import pandas as pd, statsmodels.api as sm
Y = asset_returns
X = sm.add_constant(factor_returns)
results = sm.OLS(Y, X).fit(cov_type='HAC', cov_kwds={'maxlags':5})
\end{minted}

\section{Risk‑Parity and Vol‑Target Portfolios}

Allocate weights inversely to vol or marginal contribution to risk.

\section{Kelly Criterion}

Optimal growth weight:
\(w^* = \Sigma^{-1}(\mu - r\mathbf 1)\).

Discuss leverage constraints and drawdown risk.

\section*{Key Takeaways}
\begin{enumerate}
  \item Allocation frameworks must integrate derivative Greeks for holistic risk.
  \item Kelly optimality maximises long‑run growth but increases variance.
\end{enumerate}
