\chapter{Stochastic Volatility and Jump Models}\label{ch:stochvol}

\begin{abstract}
We relax the constant‑volatility and continuous‑path assumptions.
Heston, SABR and Merton jump‑diffusion capture skew, smile and kurtosis
in observed option markets.
\end{abstract}

\section{Heston Model}

\[
\begin{aligned}
\diff S_t &= \mu S_t \diff t + \sqrt{v_t}\,S_t \diff W_t^{(1)}, \\
\diff v_t &= \kappa(\theta - v_t)\diff t + \xi\sqrt{v_t}\,\diff W_t^{(2)},\;
           \diff W^{(1)}\diff W^{(2)}=\rho\,\diff t.
\end{aligned}
\]

Characteristic‑function pricing via Fourier transform (Carr–Madan).

\section{SABR}

Forward \(F_t\) dynamics:
\[
\diff F_t = v_t F_t^\beta \diff W_t^{(1)},\quad
\diff v_t = \nu v_t \diff W_t^{(2)}.
\]

Asymptotic implied‑vol formula widely used in interest‑rate swaptions.

\section{Jump‑Diffusion (Merton)}

\(\diff S_t/S_t = (\mu-\lambda k)\diff t + \sigma \diff W_t
                  + (J-1)\diff N_t\).

PDF is infinite mixture of normals; option price is
weighted sum of Black--Scholes terms.

\section{Calibration Notes}

Non‑linear optimisation on implied‑vol surface; regularise to avoid over‑fitting.

\section*{Key Takeaways}
\begin{itemize}
  \item Stochastic vol explains smile curvature; jumps add fat tails.
  \item Semi‑closed‑form methods (Fourier, saddle‑point) aid speed.
\end{itemize}
