\chapter{Market Microstructure}\label{ch:microstructure}

\begin{abstract}
Market microstructure studies \emph{how} securities are traded:
the rules, protocols, and strategic behavior that transform latent supply
and demand into executed prices.  
Understanding these mechanics is essential for designing trading strategies,
measuring costs, and avoiding pathological market conditions such as flash crashes.
\end{abstract}

\section{Order Books and Price Formation}

Most electronic venues implement a \emph{continuous double auction (CDA)}.
Participants submit \textbf{limit orders}, which specify a price/size pair,
or \textbf{market orders}, which demand immediate execution
against the best available limits.

\begin{definition}{Limit Order Book}{}
The \emph{limit order book (LOB)} is the queue of outstanding limit orders,
ranked by price and (within the same price) by time priority.
\end{definition}

Figure~\ref{fig:lob-snapshot} shows a stylised snapshot:
\begin{center}
% \includegraphics[width=.8\textwidth]{fig/lob_snapshot.pdf}
\end{center}

\subsection{Bid–Ask Spread}

Let \(P_b\) be the highest bid and \(P_a\) the lowest ask.
The \textbf{quoted spread} is \(S_q = P_a - P_b\).
Liquidity takers implicitly pay half the spread on average.

\begin{example}{Spread Cost}{}
Suppose you buy \SI{1000}{shares} of XYZ at the ask of \$100.02
and immediately sell at the bid of \$100.00.
Your round‑trip cost is \$0.02 per share, i.e.\ \$20 total. \qedhere
\end{example}

\section{Liquidity, Depth and Resiliency}

\subsection{Instantaneous Liquidity Metrics}
Key metrics include:

\begin{itemize}
  \item \textbf{Depth} at top‑of‑book: \(\sum_{i\,:\,P_i=P_b} Q_i\)
  \item \textbf{Order‑to‑Trade Ratio}: number of displayed orders per executed share
  \item \textbf{XLM*:} smallest cost to execute a round trip of size \(Q\)
\end{itemize}

\section{Market Impact and Implementation Shortfall}

Assume your execution trajectory generates an average price
\(\bar P_{\text{exec}}\).  
The \emph{implementation shortfall} is
\(\text{IS} = (\bar P_{\text{exec}} - P_0) \times Q\),
where \(P_0\) is the decision price.

\subsection{Empirical Power Law}

Empirical studies suggest
\[
\text{Price Impact} \approx \eta \left(\frac{Q}{V}\right)^\gamma,
\quad \gamma \approx 0.5 ,
\]
where \(V\) is daily volume and \(\eta\) a stock‑specific constant
\parencite{almgren2005direct}.

\section{Order Types and Venue Features}

Beyond vanilla limits/markets, exchanges offer:

\begin{itemize}
  \item \textbf{ICEBERG}: display only a portion of true size.
  \item \textbf{PEG}: auto‑updates price relative to midpoint/primary quote.
  \item \textbf{IOC/FOK}: immediate‑or‑cancel / fill‑or‑kill time constraints.
\end{itemize}

\section{Case Study: The 2010 Flash Crash}

A large sell order in E‑mini futures interacted with
high‑frequency liquidity depletion, causing a
\SI{9}{\percent} intra‑day plunge in the S\&P 500 within minutes.
Post‑mortems highlighted the role of \emph{stop‑loss cascades}
and insufficient circuit breakers.\footnote{See CFTC/SEC joint report 2010.}

\section*{Key Takeaways}
\begin{enumerate}
  \item Execution cost decomposes into spread + impact + slippage.
  \item LOB state is \emph{state‑dependent}; supply curves are endogenous.
  \item Microstructural phenomena impose practical limits on theoretical models.
\end{enumerate}
