\chapter{Time Value of Money}\label{ch:timevalue}

\section{Discounting and Compounding}

A cash flow \(C\) at time \(T\) has present value
\(PV = C/(1+r)^T\) under annual compounding.
Continuous compounding gives \(PV = C\,e^{-rT}\).

\begin{definition}{Yield Curve}{}
A \emph{yield curve} plots term‑structure \(r(T)\).
Bootstrapping derives zero rates from coupon‑bearing bonds.
\end{definition}

\subsection{Duration and Convexity}
\[
D = -\frac{1}{P}\frac{\diff P}{\diff y},\quad
C = \frac{1}{P}\frac{\diff^2 P}{\diff y^2}.
\]

These measure first‑ and second‑order sensitivity to rate shifts.

\section{Forward Rates}
Instantaneous forward:
\(f(t) = \frac{\diff}{\diff T}\bigl(T \, r(T)\bigr)\big|_{T=t}\).

\section*{Python Example: Spot vs Forward Rates}
\begin{minted}{python}
import pandas as pd, numpy as np
times = np.array([0.5, 1, 2, 5])
spot = np.array([0.02, 0.025, 0.03, 0.035])
forward = np.diff(times * spot) / np.diff(times)
pd.DataFrame({"T_start": times[:-1], "f": forward})
\end{minted}

\section*{Key Takeaways}
\begin{enumerate}
  \item Compounding convention matters for quoted yields.
  \item DV01 and duration are linear approximations—convexity corrects curvature.
\end{enumerate}
