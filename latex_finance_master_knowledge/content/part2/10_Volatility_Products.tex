\chapter{Volatility Products}\label{ch:volprods}

\begin{abstract}
Trading \emph{volatility} directly—rather than price direction—has spawned
instruments such as variance swaps, VIX futures and vol‑target ETFs.
We formalise payoffs and common pitfalls.
\end{abstract}

\section{Implied vs Realised Volatility}

Implied \(\sigma_{\text{imp}}\) from option prices signifies a \emph{risk‑neutral}
expectation.  Realised volatility \(\sigma_{\text{real}}\) is backward‑looking.

\[% shorter equation
\text{RV}_{t,T} = \sqrt{\frac{252}{n}\sum_{i=1}^{n}\bigl(\ln S_{t_i} - \ln S_{t_{i-1}}\bigr)^2 }.
\]

\section{Variance Swaps}

Floating leg pays \(\text{RV}^2\); fixed leg is
\emph{variance strike} \(K_{\text{var}}\).
Replication requires strip of out‑of‑the‑money options:

\[
K_{\text{var}} \approx \frac{2e^{rT}}{T}
        \left[ \int_0^{F_0} \frac{P(K)}{K^2}\,\diff K
             + \int_{F_0}^{\infty} \frac{C(K)}{K^2}\,\diff K \right].
\]

\section{VIX Futures}

Cash‑settled to the 30‑day implied volatility of S\&P 500.

\subsection{Term Structure Dynamics}

Near‑term VIX often in contango except in stress regimes
where backwardation signals demand for crash protection.

\section{Volatility ETNs and Options on VIX}

Leveraged / inverse products decay via daily rebalancing mathematics.

\section*{Key Takeaways}
\begin{enumerate}
  \item Variance swaps provide pure volatility exposure, but replication
        assumes continuous hedging across strikes.
  \item Vol ETNs embed significant path‑dependent decay; position sizing
        must acknowledge compounding effects.
\end{enumerate}
