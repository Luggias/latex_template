\chapter{Options 101}\label{ch:options101}

\begin{abstract}
Options grant asymmetric payoffs via the \emph{right} but not the
obligation to transact.  We formalise terminology, payoff diagrams and
no‑arbitrage relations that underpin later pricing models.
\end{abstract}

\section{Basic Definitions}

\begin{description}
  \item[Call] Right to buy the underlying at strike \(K\).
  \item[Put]  Right to sell the underlying at strike \(K\).
  \item[European] Exercisable only at maturity.
  \item[American] Exercisable any time up to maturity.
\end{description}

\section{Payoff Diagrams}

\begin{figure}[h]
  \centering
  \begin{tikzpicture}
    \begin{axis}[axis lines=middle,xlabel={$S_T$},ylabel={Payoff},
                 domain=0:200,samples=2]
      \addplot[lw=1] {max(x-100,0)} node[pos=0.85,anchor=south east]{call};
      \addplot[lw=1] {max(100-x,0)} node[pos=0.15,anchor=north west]{put};
    \end{axis}
  \end{tikzpicture}
  \caption{European call and put payoffs at maturity.}
  \label{fig:callput}
\end{figure}

\section{Put–Call Parity}

\[
C - P = S_0 - K e^{-rT}.
\]

Immediate arbitrage if equality violated.

\section{Moneyness and Delta Approximation}

\begin{itemize}
  \item \textbf{Moneyness} \(= S_0/K\).
  \item \(\Delta_{\text{call}} \approx N(d_1)\) under Black–Scholes.
\end{itemize}

\section{Greeks Primer}

Table \ref{tab:greeks} summarises first‑order sensitivities.

\begin{table}[h]
  \centering
  \begin{tabular}{@{}lcl@{}}
  \toprule
  Greek & Symbol & Interpretation \\\midrule
  Delta & \(\Delta\) & \(\partial C/\partial S\) \\
  Gamma & \(\Gamma\) & \(\partial^2 C/\partial S^2\) \\
  Vega  & \(\nu\)    & \(\partial C/\partial \sigma\) \\
  Theta & \(\Theta\) & \(\partial C/\partial t\) \\\bottomrule
  \end{tabular}
  \caption{Primary option Greeks.}
  \label{tab:greeks}
\end{table}

\section*{Key Takeaways}
\begin{enumerate}
  \item Option structures synthesise non‑linear exposures from linear instruments.
  \item Parity and monotonicity bounds are model‑free sanity checks.
\end{enumerate}
