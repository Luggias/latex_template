\chapter{Backtesting Architecture in Python}\label{ch:backtest}

\begin{abstract}
A robust back‑tester separates market simulation, strategy logic and
broker emulation.  We implement a minimal event‑driven engine using
`pandas` and showcase pitfalls like look‑ahead bias.
\end{abstract}

\section{Core Components}

\begin{enumerate}
  \item \textbf{Event Queue} – market, signal, order, fill events.
  \item \textbf{Execution Handler} – slippage + commission model.
  \item \textbf{Portfolio} – positions, cash, realised P\&L.
\end{enumerate}

\subsection*{Skeleton Code}

\begin{minted}{python}
class MarketEvent: pass
class DataHandler:
    def get_next(self): ...
class Strategy:
    def calculate_signals(self, event): ...
class ExecutionHandler:
    def execute_order(self, order): ...
\end{minted}

\section{Vectorised vs Event‑Driven}

Vectorised (`vectorbt`, `bt`) is fast but struggles with intraday order
book logic; event‑driven (`zipline`, QuantConnect) is granular but slower.

\section{Performance Metrics}

Sharpe, Sortino, tail ratio, turnover, max drawdown, time under water.

\section*{Key Takeaways}
\begin{itemize}
  \item Decouple concerns—replace execution layer for paper‑ vs live‑trading.
  \item Always “warm up” indicators to avoid initialisation artefacts.
\end{itemize}
